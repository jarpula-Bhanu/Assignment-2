\documentclass[journal,12pt,twocolumn]{IEEEtran}
\usepackage{hyperref}
\usepackage{graphicx}
\usepackage{amsmath}
\title{Assignment 2}
\author{JARPULA BHANU PRASAD - AI21BTECH11015}
\date{Aptil 2021}
\begin{document}
\maketitle
\noindent \Large\underline{Download codes from}:
\fbox{\begin{minipage}[t]{0.95\columnwidth}
\large Download python code from - \href{https://github.com/jarpula-Bhanu/Assignment-2/blob/main/codes/function.py}{Python}\\Download latex code from - \href{https://github.com/jarpula-Bhanu/Assignment-2/blob/main/Assignment2.tex}{Latex}
\end{minipage}}

\section{\Large\underline{Problem-ICSE-2019-12 Q)4-b}}
\large\noindent Q) If $f$ : A $\rightarrow$ A and A = $R$ - \{$\frac{8}{5}$\} , show that the function $f(x)$ = $\frac{8x+3}{5x-8}$ is one-one onto. Hence, find $f^{-1}$. 
\section{\large\underline{Solution}}
\fbox{\begin{minipage}[t]{0.95\columnwidth}
\noindent \underline{Defination}: Let $f$ : $X$ $\rightarrow$ $Y$ be a function, we say $f$ is one-one or injective,\\if and only if $\forall$ $x_1$ , $x_2$ $\in$ $X$.\\ if $f(x_1)$ = $f(x_2)$ then $x_1$ = $x_2$.
\end{minipage}}
\vspace{5mm}

\noindent Suppose that $x_1$ and $x_2$ are arbitrary integers and $f(x_1)$ = $f(x_2)$, we need to show that $x_1$ = $x_2$. Since $f(x_1)$ = $f(x_2)$.\\
\begin{align*}
f(x_1)=\frac{8x_1+3}{5x_1-8} \hspace{5mm} and \hspace{5mm} f(x_2)=\frac{8x_2+3}{5x_2-8}
\end{align*}\\
\noindent Now, equating $f(x_1)$ = $f(x_2)$ since from the defination.
\begin{align} \label{1}
\implies \frac{8x_1+3}{5x_1-8} = \frac{8x_2+3}{5x_2-8}
\end{align}
On cross multiplying and simplifying the eqn\eqref{1}, we get,
\begin{align} \label{2}
\implies 49(x_1 - x_2) = 0.
\end{align}
From the eqn\eqref{2} we can say that,\\ the equation satisfies only when $x_1 - x_2$= 0. Which implies,
\begin{align*}
\Large\framebox[1.2\width]{$x_1$ = $x_2$}
\end{align*}
Hence, the given function $f(x)$ is one-one.

\vspace{5mm}
\fbox{\begin{minipage}[t]{0.95\columnwidth}
\noindent \underline{Defination}: $f$ is called onto if $f(X) = Y$.\\
i.e., Range of the function $f(x)$ is equal to co-domain.
\end{minipage}}
\vspace{5mm}

\noindent Let,
\begin{align} \label{3}
y = f(x) = \frac{8x+3}{5x-8}
\end{align}

On solving the eqn\eqref{3},\\ we get,
\begin{align} \label{4}
\LARGE\framebox[1.2\width]{$x$ = $\frac{8y+3}{5y-8}$}
\end{align}

Where $y$ is element of co-domain.\\ Now eqn\eqref{4} is defined $\forall$ $y$ $\in$ $R$ - \{$\frac{8}{5}$\}.\\ i.e.$y \in$ A.
\vspace{5mm}

\noindent now, from eqn\eqref{4} substitute the value of $x$.
\begin{align}
&\implies f(x) = f(\frac{8y+3}{5y-8}) \\
&\implies f(x) = \frac{8(\frac{8y+3}{5y-8})+3}{5(\frac{8y+3}{5y-8})-8} \\
&\implies f(x) = \frac{8(8y+3)+3(5y-8)}{5(8y+3)-8(5y-8)}\\
&\implies f(x)=\frac{79y}{79} \\
&\implies f(x) = y \label{9}
\end{align}

\noindent Hence, from eqn\eqref{9} we can say, given function $f(x)$ is onto.
\vspace{5mm}

\fbox{\begin{minipage}[t]{0.95\columnwidth}
\noindent \underline{Defination}: $f$: $X$ $\rightarrow$ $Y$ is bijective function i.e., both one-one and onto then there exit a unique function called inverse function and is denoted by $f^{-1}$, such that,
\begin{align*}
f^{-1}(y)=x \iff f(x) = y
\end{align*}
\end{minipage}}
\vspace{5mm}

\noindent Now, from defination $f^{-1}(y)$ = $x$. From eqn\eqref{4} we get the value of $x$ 

\begin{align*}
f^{-1}(y) = \frac{8y+3}{5y-8}
\end{align*}

i.e., the inverse of function $f(x)$ is,
\begin{align*}
\LARGE\framebox[1.2\width]{$f^{-1}(x)$ = $\frac{8x+3}{5x-8}$}
\end{align*}

\begin{figure}[h] 
\includegraphics[width=\columnwidth] 
{plotting}
\caption{graph}
\label{fig:a}
\end{figure}

\noindent From the above graph we observe that,\\ since the graph is continuous hence it is onto
\noindent
\fbox{\begin{minipage}[t]{0.95\columnwidth}
\noindent\small \underline{NOTE}: The above shown graph is graph of inverse function. In general for inverse function that is not one-one may have multiple images.
\end{minipage}}

\end{document}